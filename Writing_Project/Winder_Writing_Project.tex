\documentclass[12pt]{article}\usepackage[]{graphicx}\usepackage[]{color}
% maxwidth is the original width if it is less than linewidth
% otherwise use linewidth (to make sure the graphics do not exceed the margin)
\makeatletter
\def\maxwidth{ %
  \ifdim\Gin@nat@width>\linewidth
    \linewidth
  \else
    \Gin@nat@width
  \fi
}
\makeatother

\definecolor{fgcolor}{rgb}{0.345, 0.345, 0.345}
\newcommand{\hlnum}[1]{\textcolor[rgb]{0.686,0.059,0.569}{#1}}%
\newcommand{\hlstr}[1]{\textcolor[rgb]{0.192,0.494,0.8}{#1}}%
\newcommand{\hlcom}[1]{\textcolor[rgb]{0.678,0.584,0.686}{\textit{#1}}}%
\newcommand{\hlopt}[1]{\textcolor[rgb]{0,0,0}{#1}}%
\newcommand{\hlstd}[1]{\textcolor[rgb]{0.345,0.345,0.345}{#1}}%
\newcommand{\hlkwa}[1]{\textcolor[rgb]{0.161,0.373,0.58}{\textbf{#1}}}%
\newcommand{\hlkwb}[1]{\textcolor[rgb]{0.69,0.353,0.396}{#1}}%
\newcommand{\hlkwc}[1]{\textcolor[rgb]{0.333,0.667,0.333}{#1}}%
\newcommand{\hlkwd}[1]{\textcolor[rgb]{0.737,0.353,0.396}{\textbf{#1}}}%
\let\hlipl\hlkwb

\usepackage{framed}
\makeatletter
\newenvironment{kframe}{%
 \def\at@end@of@kframe{}%
 \ifinner\ifhmode%
  \def\at@end@of@kframe{\end{minipage}}%
  \begin{minipage}{\columnwidth}%
 \fi\fi%
 \def\FrameCommand##1{\hskip\@totalleftmargin \hskip-\fboxsep
 \colorbox{shadecolor}{##1}\hskip-\fboxsep
     % There is no \\@totalrightmargin, so:
     \hskip-\linewidth \hskip-\@totalleftmargin \hskip\columnwidth}%
 \MakeFramed {\advance\hsize-\width
   \@totalleftmargin\z@ \linewidth\hsize
   \@setminipage}}%
 {\par\unskip\endMakeFramed%
 \at@end@of@kframe}
\makeatother

\definecolor{shadecolor}{rgb}{.97, .97, .97}
\definecolor{messagecolor}{rgb}{0, 0, 0}
\definecolor{warningcolor}{rgb}{1, 0, 1}
\definecolor{errorcolor}{rgb}{1, 0, 0}
\newenvironment{knitrout}{}{} % an empty environment to be redefined in TeX

\usepackage{alltt}\usepackage[]{graphicx}\usepackage[]{color}
%% maxwidth is the original width if it is less than linewidth
%% otherwise use linewidth (to make sure the graphics do not exceed the margin)
\makeatletter
\def\maxwidth{ %
  \ifdim\Gin@nat@width>\linewidth
    \linewidth
  \else
    \Gin@nat@width
  \fi
}
\makeatother

\definecolor{fgcolor}{rgb}{0.345, 0.345, 0.345}
\newcommand{\hlnum}[1]{\textcolor[rgb]{0.686,0.059,0.569}{#1}}%
\newcommand{\hlstr}[1]{\textcolor[rgb]{0.192,0.494,0.8}{#1}}%
\newcommand{\hlcom}[1]{\textcolor[rgb]{0.678,0.584,0.686}{\textit{#1}}}%
\newcommand{\hlopt}[1]{\textcolor[rgb]{0,0,0}{#1}}%
\newcommand{\hlstd}[1]{\textcolor[rgb]{0.345,0.345,0.345}{#1}}%
\newcommand{\hlkwa}[1]{\textcolor[rgb]{0.161,0.373,0.58}{\textbf{#1}}}%
\newcommand{\hlkwb}[1]{\textcolor[rgb]{0.69,0.353,0.396}{#1}}%
\newcommand{\hlkwc}[1]{\textcolor[rgb]{0.333,0.667,0.333}{#1}}%
\newcommand{\hlkwd}[1]{\textcolor[rgb]{0.737,0.353,0.396}{\textbf{#1}}}%
\let\hlipl\hlkwb

\usepackage{framed}
\makeatletter
\newenvironment{kframe}{%
 \def\at@end@of@kframe{}%
 \ifinner\ifhmode%
  \def\at@end@of@kframe{\end{minipage}}%
  \begin{minipage}{\columnwidth}%
 \fi\fi%
 \def\FrameCommand##1{\hskip\@totalleftmargin \hskip-\fboxsep
 \colorbox{shadecolor}{##1}\hskip-\fboxsep
     % There is no \\@totalrightmargin, so:
     \hskip-\linewidth \hskip-\@totalleftmargin \hskip\columnwidth}%
 \MakeFramed {\advance\hsize-\width
   \@totalleftmargin\z@ \linewidth\hsize
   \@setminipage}}%
 {\par\unskip\endMakeFramed%
 \at@end@of@kframe}
\makeatother

\definecolor{shadecolor}{rgb}{.97, .97, .97}
\definecolor{messagecolor}{rgb}{0, 0, 0}
\definecolor{warningcolor}{rgb}{1, 0, 1}
\definecolor{errorcolor}{rgb}{1, 0, 0}
\newenvironment{knitrout}{}{} % an empty environment to be redefined in TeX

\usepackage{alltt}
\usepackage[english]{babel}
\usepackage[utf8]{inputenc}
\usepackage{amsmath}
\usepackage{graphicx}
%\usepackage{cite}
\usepackage{url}
\usepackage{caption}
\usepackage{setspace}
\IfFileExists{upquote.sty}{\usepackage{upquote}}{}
\usepackage{listings}
\usepackage{amsthm}
\usepackage[linewidth=1pt]{mdframed}
\usepackage{lipsum}
%\usepackage{natbib}
%\usepackage[round]{natbib} 
\usepackage{apalike}
\lstset{
	language=R,
	basicstyle=\ttfamily
}
\usepackage{epsfig} % eps graphics
%authoryear,open={(},close={)}}
\IfFileExists{upquote.sty}{\usepackage{upquote}}{}
\begin{document}

\begin{titlepage}

\newcommand{\HRule}{\rule{\linewidth}{0.5mm}} % Defines a new command for the horizontal lines, change thickness here

\center % Center everything on the page
 
%----------------------------------------------------------------------------------------
%   HEADING SECTIONS
%----------------------------------------------------------------------------------------

\textsc{\LARGE Montana State University}\\[0.5cm] % Name of your university/college
\textsc{\Large Department of Mathematical Sciences}\\[0.5cm] % Major heading such as course name
\textsc{\large Writing Project}\\[.75cm] % Minor heading such as course title

%----------------------------------------------------------------------------------------
%   TITLE SECTION
%----------------------------------------------------------------------------------------

\HRule \\[0.4cm]
{ \huge \bfseries TITLE }\\[0.4cm] % Title of your document
\HRule \\[1.5cm]
 
%----------------------------------------------------------------------------------------
%   AUTHOR SECTION
%----------------------------------------------------------------------------------------

\begin{minipage}{0.4\textwidth}
\begin{flushleft} 
\large
\emph{Author:} \\
Meaghan \textsc{Winder} \\
\end{flushleft}
\end{minipage}
~
\begin{minipage}{0.4\textwidth}
\begin{flushright} 
\large
\emph{Supervisor:} \\
Dr. Andrew \textsc{Hoegh} 
\end{flushright}
\end{minipage}\\[1.5cm]

%----------------------------------------------------------------------------------------
%   DATE SECTION
%----------------------------------------------------------------------------------------

{\large Spring 2020}\\[1.5cm] % Date, change the \today to a set date if you want to be precise

%----------------------------------------------------------------------------------------
%   LOGO SECTION
%----------------------------------------------------------------------------------------

\includegraphics[width=5cm]{msulogo} % Include a department/university logo - this will require the graphicx package
 
%----------------------------------------------------------------------------------------

A writing project submitted in partial fulfillment\\
of the requirements for the degree\\[.25in]
Master's of Science in Statistics

\vfill % Fill the rest of the page with whitespace

\end{titlepage}

\begin{titlepage}
\null
\begin{center}
{\bf\huge APPROVAL}\\[1.0 in]
of a writing project submitted by\\[.25 in]
Meaghan Winder \\[0.5 in]
\end{center}

\noindent
This writing project has been read by the writing project advisor and
has been found to be satisfactory regarding content, English usage,
format, citations, bibliographic style, and consistency, and is ready
for submission to the Statistics Faculty.

\vspace{.3in}
\begin{center}
\begin{tabular}{ll}
\rule{2.75in}{.03in} & \rule{2.75in}{.03in} \\
Date& Andrew Hoegh \\
& Writing Project Advisor \\
\end{tabular}
\end{center}

\vspace{1cm}

\begin{center}
\begin{tabular}{ll}
\rule{2.75in}{.03in} & \rule{2.75in}{.03in} \\
Date& Mark C. Greenwood \\
& Writing Projects Coordinator \\
\end{tabular}
\end{center}

\end{titlepage}

\newpage
\tableofcontents
\newpage

\begin{abstract}
\noindent abstract text here 
\end{abstract}

\doublespacing

\noindent{\bfseries free write... this is very messy, basically just word vomit}

\noindent{\bfseries This is all information extracted from WILD 502 course material or things I learned from the class} 

Occupancy is the presence of a particular species on a given site, this may not be the first choice of state variables to ecologists but occupancy studies are useful when there is a large spatial scale or the study is conducted over many years, when abundance or vital rates are hard to measure. Occupancy studies are also useful over capture-recapture methods when individuals cannot be marked or uniquely identified. However, sometimes patterns of species occurrence are of interest, this happens when researchers are interested in the range of a species or the spread of invasion. 

The sampling units for occupancy studies are called 'sites'. We can learn about detection probabilities when multiple site visits are used. Also, when using occupancy models we need to account for imperfect detection because it is possible that the researchers could miss the species even if it is present at the site. $\psi$ represents the occupancy probability, $p_i$ represents the probability of detecting the species on survey $i$ given that the species occupies the site, and $p^* = 1- \prod_{i = 1}^{t} (1-p_i)$ is the probability of detecting the species at least one time given the species occupies the site. 

The assumptions are: 
\begin{itemize}
\item The occupancy state of sites is constant during a single season. 
\item The occupancy probability is constant across sites, or is modeled appropriately using site-level covariates. 
\item The probability of detection given occupancy status is constant across sites, or modeled appropriately using site-level covariates. 
\item The species is not misidentified, no false positives. 
\end{itemize}

As suggested above, site-level covariates can be used to model the occupancy probabilities and the detection probabilities. 

In WILD 502 when talking about multi-season occupancy models, we talked about extirpation and colonization rates, but I think that these could be modeled with a latent variable(s)? I don't think they are of particular interest here. 

In this case, if we consider site to be the lake then we have replication at the site level, but if each site is the sample location within the lake then there is replication for some (very few) on different dates. 

\noindent{\bfseries This is what I used for my project proposal... Mark said this could make it's way into our abstract}

Zebra mussels are a highly invasive species that have several negative impacts, both ecologically and economically, as they suffocate native mussels and cost millions of dollars to remove from man-made structures such as power plants (USGS). Zebra mussels have been established in the United States since the 1980s, and since then "they have spread rapidly throughout the Great Lakes region and into the large rivers of the eastern Mississippi drainage. They have also been found in Texas, Colorado, Utah, Nevada, and California" (USGS). As the threat of zebra mussels spreading into the northwestern United States grows, one concern is that the detection method used to find zebra mussels is not highly effective. The current method of sampling are plankton tows, which are essentially large nets with fine mesh, that are towed behind boats to capture zebra mussel veligers (larval stage of zebra mussels); this method only works to capture veligers because once the zebra mussels mature, they attach to substrate and are no longer able to be captured by the tows. Researchers typically go to several sites on a given day within a water body and take multiple plankton tows (usually 5). The contents of the tows is aggregated for each site, then taken to a lab, where scientists use a microscope to examine the contents for presence (sometimes counts) of zebra mussel veligers, if the veligers are not detected with the microscope, the scientists use Polymerase chain reaction (PCR) to  test for presence of zebra mussels veligers. A large problem with this method, is that there is a non-zero probability that the zebra mussels are in the water body and not captured with the tows, or in the tows and not identified with the microscope or with PCR. 

\noindent{\bfseries I have also been working on fitting models to simulated occupancy data} (see code at \url{https://github.com/meaghanwinder/Writing-Project/blob/master/Writing%20Project%20Code.Rmd})

\begin{itemize}
\item constant $\psi = 0.6$ and constant $p = 0.6$ with constant $J = 8$ samples per $M = 10$ sites -- DONE
\item constant $\psi = 0.6$ and constant $p = 0.6$ with $j_i$ samples per $M = 10$ sites -- DONE
\item $\psi$ and $p$ both depend on a single site level covariate, $x$, with $j_i$ samples per $M = 10$ sites
\end{itemize}

$y_{i,j} = \{0, 1\}$ is the binary response for the $j^{th}$ sample within site $i$

If the site is occupied: 
\begin{itemize}
\item $y_{i,j} \sim Bernoulli(p)$
\item $y_i \sim Binomial(J, p)$ where $y_i$ is the total number of detections from the $J$ samples in the $i^{th}$ site 
\end{itemize}

If the site is unoccupied: 
\begin{itemize}
\item $y_{i,j} = 0$ with probability 1, since we assume there are no false detections
\end{itemize}







Things to think about for the simulation or questions I have: 
\begin{itemize}
\item think about how changing the detection probabilities and occupancy probabilities impact the results: 
	\begin{itemize}
	\item High occupancy, high detection
	\item High occupancy, low detection 
	\item Low occupancy, high detection 
	\item Low occupancy, low detection 
	\end{itemize}
\item How do we/can we include sample level covariates to account for sampling effort (number of tows, if available)?
\item If we are getting more data:
	\begin{itemize}
	\item How do we account for multiple sampling seasons (years)? 
	\item How do the assumptions of occupancy models change when we have several sampling years versus only 1
\end{itemize}
\end{itemize}

Additional questions, not directly related to the simulation:
\begin{itemize}
	\item How do we account for this multilevel (for lack of a better word) testing process? For example, sometimes (?) when they fail to find them in the microscope, then they test them using polymerase chain reaction (PCR), and sometimes gene sequencing (? -- looks like it was only used on one observation in the sample data), or scanning electron microscopy (SEM).
	\item One assumption of occupancy models is that there is a non-zero probability of mis-identification. I read an article (Denise Holser) that suggests that there might be an issue with that assumption in this sitatuion because there are similar looking organisms that may be present in the waters; she then suggest that the Bureau of Reclaimation has attempted to mediate this issue with improved microscopic methods and improved PCR methods. I will look into this some more.   
\end{itemize}


\section{Introduction}



One criticism of the detection of zebra mussels using eDNA is that there is a non-zero false positive rate (this has been worked on?), what is the false negative rate?
\begin{itemize}
\item ``a positive eDNA detection can occur even though the living target taxa was never present in the sampled water body'' \cite{Sepulveda:eDNA}
\end{itemize}

plankton tow methods: 
\begin{itemize}
\item ``Plankton tow sampling is the current standard for early detection of dressenids in western North America. Using a net, large amounts of water and debris are collected and then concentrated for microscopic examination of the free-swimming larval form (i.e., veligers) of dreissenid mussels.'' identification is confirmed with a DNA test (PCR) \cite{Sepulvelda:eDNA} 
\end{itemize}

Research question(s): How does the detection probability of zebra mussels using microscopy compare to the detection probability of zebra mussels using eDNA? What is the false negative rate for both methods? Given that, how many samples should we collect at a given waterbody? 


\section{Data}

This could maybe be included in the introduction section... unless I do EDA here, then probably leave this as its own section



\section{Modeling Background}

This whole section might be able to be merged with the methods section 

\subsection{Occupancy Models}
 
\subsection{Bayesian Modeling Background}

\section{Methods}

Package options for Multi-season single-species occupancy models: 
\begin{itemize}
\item nimble.dynamic.occ
\item STAN
\item JAGS
\item wiquid package??
\item Frequentist Options: 
  \begin{itemize}
  \item unmarked
  \item Program MARK
  \end{itemize}
\item write my own package? 
\end{itemize}

\section{Analysis}

Could do EDA here

\section{Conclusion}

\subsection{Further Investigations}

\newpage
\section{References}
\begingroup
\renewcommand{\section}[2]{}%
\begin{flushleft}
\bibliographystyle{apalike}
%\bibliographystyle{acm}
%\bibliographystyle{abbrvnat}
%\bibliographystyle{unsrtnat}
%\bibliographystyle{ACM-Reference-Format}
\bibliography{bibliography}
\end{flushleft}
\endgroup

\newpage
\section{Appendix - R Code}

A script containing all code used for this analysis is available at \\
\centering{\textit{github link here(?)... either that or include all code here}} 

\singlespacing

\begin{knitrout}
\definecolor{shadecolor}{rgb}{0.969, 0.969, 0.969}\color{fgcolor}\begin{kframe}
\begin{alltt}
\hlcom{# SIMULATED DATA 1}
\hlkwd{set.seed}\hlstd{(}\hlnum{1202020}\hlstd{)}
\hlstd{p} \hlkwb{<-} \hlnum{0.6} \hlcom{# constant detection probability... could change to depend on covariates}
\hlstd{psi} \hlkwb{<-} \hlnum{0.6} \hlcom{# constant occupancy probability... could change to depend on covariates }
\hlstd{M} \hlkwb{<-} \hlnum{10} \hlcom{# number of sites}
\hlstd{J} \hlkwb{<-} \hlnum{8} \hlcom{# constant number of samples per site... does not need to be constant}

\hlstd{z} \hlkwb{<-} \hlkwd{rbinom}\hlstd{(M,} \hlnum{1}\hlstd{, psi)} \hlcom{# site occupancy}

\hlstd{y1} \hlkwb{<-} \hlkwd{matrix}\hlstd{(}\hlnum{NA}\hlstd{,} \hlkwc{nrow} \hlstd{= M}\hlopt{*}\hlstd{J,} \hlkwc{ncol} \hlstd{=} \hlnum{4}\hlstd{)}
\hlstd{y1[,} \hlnum{1}\hlstd{]} \hlkwb{<-} \hlkwd{rep}\hlstd{(}\hlnum{1}\hlopt{:}\hlstd{M,} \hlkwc{each}  \hlstd{= J)} \hlcom{# column indicating site}
\hlstd{y1[,} \hlnum{2}\hlstd{]} \hlkwb{<-} \hlkwd{rep}\hlstd{(}\hlnum{1}\hlopt{:}\hlstd{J, M)} \hlcom{# column indicating sample within site}
\hlstd{y1[,} \hlnum{3}\hlstd{]} \hlkwb{<-} \hlkwd{rep}\hlstd{(z,} \hlkwc{each} \hlstd{= J)} \hlcom{# column indicating true site occupancy}

\hlcom{# creates a column of 1's and 0's indicating whether the species was detected}
\hlkwa{for}\hlstd{(i} \hlkwa{in} \hlnum{1}\hlopt{:}\hlstd{(M}\hlopt{*}\hlstd{J))\{}
    \hlstd{y1[i,} \hlnum{4}\hlstd{]} \hlkwb{<-} \hlkwd{rbinom}\hlstd{(}\hlnum{1}\hlstd{,} \hlnum{1} \hlstd{, p}\hlopt{*}\hlstd{y1[i,} \hlnum{3}\hlstd{])} \hlcom{# if the site is not occupied the species cannot be detected}
\hlstd{\}}

\hlkwd{colnames}\hlstd{(y1)} \hlkwb{<-} \hlkwd{c}\hlstd{(}\hlstr{"Site"}\hlstd{,} \hlstr{"Sample"}\hlstd{,} \hlstr{"True Occupancy"}\hlstd{,} \hlstr{"Detected"}\hlstd{)}

\hlkwd{library}\hlstd{(}\hlstr{"car"}\hlstd{)}
\hlkwd{some}\hlstd{(y1)} \hlcom{# view 10 sample rows of the simulated data}

\hlcom{# or}

\hlstd{y2} \hlkwb{<-} \hlkwd{rbinom}\hlstd{(M, J, p}\hlopt{*}\hlstd{z)} \hlcom{# total number of detections for the J samples within each of the M sites, y_i}
\hlcom{# SIMULATED DATA 2}
\hlkwd{set.seed}\hlstd{(}\hlnum{1222020}\hlstd{)}
\hlstd{p} \hlkwb{<-} \hlnum{0.6} \hlcom{# constant detection probability... could change to depend on covariates}
\hlstd{psi} \hlkwb{<-} \hlnum{0.6} \hlcom{# constant occupancy probability... could change to depend on covariates }
\hlstd{M} \hlkwb{<-} \hlnum{10} \hlcom{# number of sites}
\hlstd{J} \hlkwb{<-} \hlkwd{sample}\hlstd{(}\hlnum{1}\hlopt{:}\hlnum{10}\hlstd{, M,} \hlkwc{replace} \hlstd{= T)} \hlcom{# number of times each of the sites were sampled}

\hlstd{z} \hlkwb{<-} \hlkwd{rbinom}\hlstd{(M,} \hlnum{1}\hlstd{, psi)} \hlcom{# site occupancy}

\hlstd{y} \hlkwb{<-} \hlkwd{rep}\hlstd{(}\hlnum{NA}\hlstd{, M)}

\hlkwa{for}\hlstd{(i} \hlkwa{in} \hlnum{1}\hlopt{:}\hlstd{M)\{}
  \hlstd{y[i]} \hlkwb{<-} \hlkwd{rbinom}\hlstd{(}\hlnum{1}\hlstd{, J[i], p}\hlopt{*}\hlstd{z[i])} \hlcom{# total number of detections for the J samples within each of the M sites, y_i}
\hlstd{\}}
\hlcom{#SIMULATED DATA 3}
\hlkwd{set.seed}\hlstd{(}\hlnum{1222020}\hlstd{)}
\hlstd{M} \hlkwb{<-} \hlnum{10} \hlcom{# number of sites}
\hlstd{x} \hlkwb{<-} \hlkwd{runif}\hlstd{(}\hlnum{10}\hlstd{,} \hlnum{0}\hlstd{,} \hlnum{10}\hlstd{)}
\hlstd{beta1.true} \hlkwb{<-} \hlnum{2}
\hlstd{beta2.true} \hlkwb{<-} \hlnum{0.5}
\hlstd{beta3.true} \hlkwb{<-} \hlnum{0.5}
\hlstd{beta4.true} \hlkwb{<-} \hlnum{0.2}
\hlstd{p} \hlkwb{<-} \hlkwd{exp}\hlstd{(beta1.true} \hlopt{-} \hlstd{beta2.true}\hlopt{*}\hlstd{x)}\hlopt{/}\hlstd{(}\hlnum{1} \hlopt{+} \hlkwd{exp}\hlstd{(beta1.true} \hlopt{-} \hlstd{beta2.true}\hlopt{*}\hlstd{x))} \hlcom{# detection probability depends on site covariate x}
\hlstd{psi} \hlkwb{<-} \hlkwd{exp}\hlstd{(beta3.true} \hlopt{+} \hlstd{beta4.true}\hlopt{*}\hlstd{x)}\hlopt{/}\hlstd{(}\hlnum{1} \hlopt{+} \hlkwd{exp}\hlstd{(beta3.true} \hlopt{+} \hlstd{beta4.true}\hlopt{*}\hlstd{x))} \hlcom{# occupancy probability depends on site covariate x}
\hlstd{J} \hlkwb{<-} \hlkwd{sample}\hlstd{(}\hlnum{1}\hlopt{:}\hlnum{10}\hlstd{, M,} \hlkwc{replace} \hlstd{= T)} \hlcom{# number of times each of the sites were sampled}

\hlstd{z} \hlkwb{<-} \hlkwd{rep}\hlstd{(}\hlnum{NA}\hlstd{, M)}
\hlstd{y} \hlkwb{<-} \hlkwd{rep}\hlstd{(}\hlnum{NA}\hlstd{, M)}

\hlkwa{for}\hlstd{(i} \hlkwa{in} \hlnum{1}\hlopt{:}\hlstd{M)\{}
  \hlstd{z[i]} \hlkwb{<-} \hlkwd{rbinom}\hlstd{(}\hlnum{1}\hlstd{,} \hlnum{1}\hlstd{, psi[i])} \hlcom{#site occupancy}
  \hlstd{y[i]} \hlkwb{<-} \hlkwd{rbinom}\hlstd{(}\hlnum{1}\hlstd{, J[i], p[i]}\hlopt{*}\hlstd{z[i])} \hlcom{# total number of detections for the J samples within each of the M sites, y_i}
\hlstd{\}}
\hlkwd{library}\hlstd{(}\hlstr{"readxl"}\hlstd{)}
\hlstd{USBR.login} \hlkwb{<-} \hlkwd{read_excel}\hlstd{(}\hlstr{"C:/Users/mwind/OneDrive/Writing-Project/USBR_2018_data_sample.xlsx"}\hlstd{,} \hlkwc{sheet} \hlstd{=} \hlnum{1}\hlstd{)}
\hlstd{USBR.sampleprep} \hlkwb{<-} \hlkwd{read_excel}\hlstd{(}\hlstr{"C:/Users/mwind/OneDrive/Writing-Project/USBR_2018_data_sample.xlsx"}\hlstd{,} \hlkwc{sheet} \hlstd{=} \hlnum{2}\hlstd{)}
\hlstd{USBR.microscopy} \hlkwb{<-} \hlkwd{read_excel}\hlstd{(}\hlstr{"C:/Users/mwind/OneDrive/Writing-Project/USBR_2018_data_sample.xlsx"}\hlstd{,} \hlkwc{sheet} \hlstd{=} \hlnum{3}\hlstd{)}
\hlstd{USBR.PCR} \hlkwb{<-} \hlkwd{read_excel}\hlstd{(}\hlstr{"C:/Users/mwind/OneDrive/Writing-Project/USBR_2018_data_sample.xlsx"}\hlstd{,} \hlkwc{sheet} \hlstd{=} \hlnum{4}\hlstd{)}

\hlkwd{unique}\hlstd{(USBR.sampleprep}\hlopt{$}\hlstd{`Water Name`)} \hlcom{# number of lakes (sites) in the 2018 sample data}
\end{alltt}
\end{kframe}
\end{knitrout}


\lstset{
	basicstyle=\ttfamily\scriptsize,
	columns=fullflexible,
	frame=single,
	breaklines=true,
	postbreak=\mbox{\textcolor{red}{$\hookrightarrow$}\space},
	keepspaces = TRUE,
}
%\begin{lstlisting}
%\end{lstlisting}
\end{document}
              
