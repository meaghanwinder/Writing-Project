\documentclass[12pt]{article}\usepackage[]{graphicx}\usepackage[]{color}
% maxwidth is the original width if it is less than linewidth
% otherwise use linewidth (to make sure the graphics do not exceed the margin)
\makeatletter
\def\maxwidth{ %
  \ifdim\Gin@nat@width>\linewidth
    \linewidth
  \else
    \Gin@nat@width
  \fi
}
\makeatother

\definecolor{fgcolor}{rgb}{0.345, 0.345, 0.345}
\newcommand{\hlnum}[1]{\textcolor[rgb]{0.686,0.059,0.569}{#1}}%
\newcommand{\hlstr}[1]{\textcolor[rgb]{0.192,0.494,0.8}{#1}}%
\newcommand{\hlcom}[1]{\textcolor[rgb]{0.678,0.584,0.686}{\textit{#1}}}%
\newcommand{\hlopt}[1]{\textcolor[rgb]{0,0,0}{#1}}%
\newcommand{\hlstd}[1]{\textcolor[rgb]{0.345,0.345,0.345}{#1}}%
\newcommand{\hlkwa}[1]{\textcolor[rgb]{0.161,0.373,0.58}{\textbf{#1}}}%
\newcommand{\hlkwb}[1]{\textcolor[rgb]{0.69,0.353,0.396}{#1}}%
\newcommand{\hlkwc}[1]{\textcolor[rgb]{0.333,0.667,0.333}{#1}}%
\newcommand{\hlkwd}[1]{\textcolor[rgb]{0.737,0.353,0.396}{\textbf{#1}}}%
\let\hlipl\hlkwb

\usepackage{framed}
\makeatletter
\newenvironment{kframe}{%
 \def\at@end@of@kframe{}%
 \ifinner\ifhmode%
  \def\at@end@of@kframe{\end{minipage}}%
  \begin{minipage}{\columnwidth}%
 \fi\fi%
 \def\FrameCommand##1{\hskip\@totalleftmargin \hskip-\fboxsep
 \colorbox{shadecolor}{##1}\hskip-\fboxsep
     % There is no \\@totalrightmargin, so:
     \hskip-\linewidth \hskip-\@totalleftmargin \hskip\columnwidth}%
 \MakeFramed {\advance\hsize-\width
   \@totalleftmargin\z@ \linewidth\hsize
   \@setminipage}}%
 {\par\unskip\endMakeFramed%
 \at@end@of@kframe}
\makeatother

\definecolor{shadecolor}{rgb}{.97, .97, .97}
\definecolor{messagecolor}{rgb}{0, 0, 0}
\definecolor{warningcolor}{rgb}{1, 0, 1}
\definecolor{errorcolor}{rgb}{1, 0, 0}
\newenvironment{knitrout}{}{} % an empty environment to be redefined in TeX

\usepackage{alltt}\usepackage[]{graphicx}\usepackage[]{color}
%% maxwidth is the original width if it is less than linewidth
%% otherwise use linewidth (to make sure the graphics do not exceed the margin)
\makeatletter
\def\maxwidth{ %
  \ifdim\Gin@nat@width>\linewidth
    \linewidth
  \else
    \Gin@nat@width
  \fi
}
\makeatother

\definecolor{fgcolor}{rgb}{0.345, 0.345, 0.345}
\newcommand{\hlnum}[1]{\textcolor[rgb]{0.686,0.059,0.569}{#1}}%
\newcommand{\hlstr}[1]{\textcolor[rgb]{0.192,0.494,0.8}{#1}}%
\newcommand{\hlcom}[1]{\textcolor[rgb]{0.678,0.584,0.686}{\textit{#1}}}%
\newcommand{\hlopt}[1]{\textcolor[rgb]{0,0,0}{#1}}%
\newcommand{\hlstd}[1]{\textcolor[rgb]{0.345,0.345,0.345}{#1}}%
\newcommand{\hlkwa}[1]{\textcolor[rgb]{0.161,0.373,0.58}{\textbf{#1}}}%
\newcommand{\hlkwb}[1]{\textcolor[rgb]{0.69,0.353,0.396}{#1}}%
\newcommand{\hlkwc}[1]{\textcolor[rgb]{0.333,0.667,0.333}{#1}}%
\newcommand{\hlkwd}[1]{\textcolor[rgb]{0.737,0.353,0.396}{\textbf{#1}}}%
\let\hlipl\hlkwb

\usepackage{framed}
\makeatletter
\newenvironment{kframe}{%
 \def\at@end@of@kframe{}%
 \ifinner\ifhmode%
  \def\at@end@of@kframe{\end{minipage}}%
  \begin{minipage}{\columnwidth}%
 \fi\fi%
 \def\FrameCommand##1{\hskip\@totalleftmargin \hskip-\fboxsep
 \colorbox{shadecolor}{##1}\hskip-\fboxsep
     % There is no \\@totalrightmargin, so:
     \hskip-\linewidth \hskip-\@totalleftmargin \hskip\columnwidth}%
 \MakeFramed {\advance\hsize-\width
   \@totalleftmargin\z@ \linewidth\hsize
   \@setminipage}}%
 {\par\unskip\endMakeFramed%
 \at@end@of@kframe}
\makeatother

\definecolor{shadecolor}{rgb}{.97, .97, .97}
\definecolor{messagecolor}{rgb}{0, 0, 0}
\definecolor{warningcolor}{rgb}{1, 0, 1}
\definecolor{errorcolor}{rgb}{1, 0, 0}
\newenvironment{knitrout}{}{} % an empty environment to be redefined in TeX

\usepackage{alltt}
\usepackage[english]{babel}
\usepackage[utf8]{inputenc}
\usepackage{amsmath}
\usepackage{graphicx}
\usepackage{cite}
\usepackage{url}
\usepackage{caption}
\usepackage{setspace}
\IfFileExists{upquote.sty}{\usepackage{upquote}}{}
\usepackage{listings}
\usepackage{amsthm}
\usepackage[linewidth=1pt]{mdframed}
\usepackage{lipsum}
%\usepackage{natbib}
%\usepackage[round]{natbib} 
\usepackage{apalike}
\lstset{
	language=R,
	basicstyle=\ttfamily
}
\usepackage{epsfig} % eps graphics
%authoryear,open={(},close={)}}
\IfFileExists{upquote.sty}{\usepackage{upquote}}{}
\begin{document}

\begin{titlepage}

\newcommand{\HRule}{\rule{\linewidth}{0.5mm}} % Defines a new command for the horizontal lines, change thickness here

\center % Center everything on the page
 
%----------------------------------------------------------------------------------------
%   HEADING SECTIONS
%----------------------------------------------------------------------------------------

\textsc{\LARGE Montana State University}\\[0.5cm] % Name of your university/college
\textsc{\Large Department of Mathematical Sciences}\\[0.5cm] % Major heading such as course name
\textsc{\large Writing Project}\\[.75cm] % Minor heading such as course title

%----------------------------------------------------------------------------------------
%   TITLE SECTION
%----------------------------------------------------------------------------------------

\HRule \\[0.4cm]
{ \huge \bfseries TITLE }\\[0.4cm] % Title of your document
\HRule \\[1.5cm]
 
%----------------------------------------------------------------------------------------
%   AUTHOR SECTION
%----------------------------------------------------------------------------------------

\begin{minipage}{0.4\textwidth}
\begin{flushleft} 
\large
\emph{Author:} \\
\textsc{Meaghan Winder} \\
\end{flushleft}
\end{minipage}
~
\begin{minipage}{0.4\textwidth}
\begin{flushright} 
\large
\emph{Supervisor:} \\
\textsc{Dr. Andrew Hoegh} 
\end{flushright}
\end{minipage}\\[1.5cm]

%----------------------------------------------------------------------------------------
%   DATE SECTION
%----------------------------------------------------------------------------------------

{\large Spring 2020}\\[1.5cm] % Date, change the \today to a set date if you want to be precise

%----------------------------------------------------------------------------------------
%   LOGO SECTION
%----------------------------------------------------------------------------------------

\includegraphics[width=5cm]{msulogo} % Include a department/university logo - this will require the graphicx package
 
%----------------------------------------------------------------------------------------

A writing project submitted in partial fulfillment\\
of the requirements for the degree\\[.25in]
Master's of Science in Statistics

\vfill % Fill the rest of the page with whitespace

\end{titlepage}

\begin{titlepage}
\null
\begin{center}
{\bf\huge APPROVAL}\\[1.0 in]
of a writing project submitted by\\[.25 in]
Meaghan Winder \\[0.5 in]
\end{center}

\noindent
This writing project has been read by the writing project advisor and
has been found to be satisfactory regarding content, English usage,
format, citations, bibliographic style, and consistency, and is ready
for submission to the Statistics Faculty.

\vspace{.3in}
\begin{center}
\begin{tabular}{ll}
\rule{2.75in}{.03in} & \rule{2.75in}{.03in} \\
Date& Andrew Hoegh \\
& Writing Project Advisor \\
\end{tabular}
\end{center}

\vspace{1cm}

\begin{center}
\begin{tabular}{ll}
\rule{2.75in}{.03in} & \rule{2.75in}{.03in} \\
Date& Mark C. Greenwood \\
& Writing Projects Coordinator \\
\end{tabular}
\end{center}

\end{titlepage}

\newpage
\tableofcontents
\newpage

\begin{abstract}
\noindent abstract text here 
\end{abstract}

\doublespacing

\section{Introduction}

In early 2020, the City of Austin, Texas approved the spending of four million dollars over the next five years in an attempt to remove zebra mussels from the city's source of drinking water with a liquid copper sulfate pentahydrate released into the water intake pipes \cite{CBS:Austin}. This is one of many pursuits to remove dreissenid mussels\footnote{Zebra mussels (\textit{Dreissena polymorpha}) and quagga mussels (\textit{Dreissena rostriformis bugensis}) collectively.} from water bodies across the United States, and four million dollars is only a small fraction of what is spent annually on control and mitigation efforts. 

Zebra mussels are native to the Caspian and Black Seas, but have become widespread in both Europe and the United States; they were discovered in the Great Lakes in the late 1980s and have since spread rapidly across the United States. The United States National Park Service stated that ``[o]nce a population of zebra mussels has become established in a water body, there is very little to be done to remove them. Prevention, therefore, is the best way to keep a water body clean of zebra mussels'' \cite{NPS}; hence, early detection of invasive species, such as dreissenid mussels, has become a priority, so that organizations can plan, budget, and install necessary technologies before colonization has occured \cite{Holser:body}. 

Zebra mussels live between two and five years; they start as microscopic veligers but mature to thumbnail sized adults; they begin reproduction at two years of age, after which, females can release up to one million eggs per year \cite{NPS}. Dreissenid mussel veligers free-swim in the water; often, they travel to uninfested waters on boats or through other aquatic recreational activies, however, sometimes they are moved by nature and travel downstream to uninfested waters. Adult dreissenids attach and colonize hard surfaces in the water, which often results in costly removal procedures \cite{Holser:detection}. This process of accumulation of adult zebra mussels on rocks, native mussels, docks, boats, or other hard surfaces is referred to as ``biofouling,'' and objects that are in the water for long periods of time become difficult and costly to clean. Water supply and delivery facilities, water recreation sites, and other water dependent economies in dreissenid mussel infested waters become much more expensive to maintain and operate \cite{BOR}. Dreissenid infestations result not only in economic impacts, but in environmental ones as well. Dreissenid mussels are filter feeders and siphon plankton from the water, which can lead to changes the water body ecosystem by increasing water clarity; a single adult dreissenid can filter about a liter of water per day, which reduces the availability of algae for native mussels and bottom feeding fish \cite{BOR}. Additionally, ``biofouling'' can prevent native mussels from moving, feeding, reproducing, or regulating the water system. Several actions, such as the 2017 initiative, \textit{Safeguarding the West from Invasive Species}, by the Department of the Interior, have been taken to protect water bodies in the western United States from the economic and ecological threats posed by the invasive dreissenid mussels. Early detection of dreissenid mussel species can reduce the economic and ecological reprocussions of dreissenid infestations, however there are issues with the available early detection methods. 

The established standard for early detection of dreissenids in the western United States is plankton tow sampling for mussel veligers. Using a fine mesh net, water and debris are collected at multiple sampling sites within each water body; the debris from each net collected at the same sampling site on the same day is aggregated and examined, using cross-polarized light microscopy, for the free-swimming veligers. Following the microscopic examination, positive species identification is confirmed using polymerase chain reaction (PCR). This early detection method requires a breeding population, so is limited to the weeks immediately following a spawning event \cite{Nichols}; spawning begins at water temperatures above 10$^\circ$ C for quagga mussels and above 12$^\circ$ C for zebra mussels \cite{McMahon, Mills}. This suggests that veliger availability in northern latitude water bodies is typically limited to warmer months \cite{Sepulveda:eDNA}.

An alternative method for detection of rare, endangered, or invasive aquatic organisms, one growing in popularity, are environmental DNA (eDNA) surveys \cite{Schmelzle}. Environmental DNA methods can detect DNA diffused from the target species from water sampled from a water body. Sepulveda, Amberg, and Hanson suggest the use of eDNA surveys may widen the seasonal sampling window over plankton tow methods, since eDNA does not rely on a breeding population (2019). This method is more time and cost effective than traditional sampling methods for species of low abundance \cite{Rees}. However, a positive eDNA result does not necessarily mean the target species is present or alive at the site; positive eDNA results can be obtained from ``a failed introduction, from external sources, or from field contamination, rather than fresh DNA from mussel colonization'' \cite{Sepulveda:eDNA}. One criticism of the detection of dreissenid mussels using eDNA is there is a possibility of obtaining false-positive results; since control efforts for invasive species are costly, there is some hesitation in using eDNA surveys as the sole decision-making tool for the management of invasive species. 

When these two methods result in conflicting answers, decision making can be even more complicated, since a positive eDNA result only suggests that the DNA of the target species is present, regardless of whether the species is alive or even present at all, but when veligers are detected, positive eDNA results indicate a potential colonization, which is useful to managers \cite{Holser:body}.


\textbf{I wrote all this, and then realized I should probably include at least some stuff about occupancy models in order to discuss the research questions. I will work on incorporating this more when we nail down the data sets for sure. Also, the research questions might need some tweaking based on which data set we do, or if we do both.} 

Research question(s): How does the detection probability of zebra mussels using microscopy compare to the detection probability of zebra mussels using eDNA? What is the false negative rate for both methods? Given that, how many samples should we collect at a given water body? 


\section{Data}

This could maybe be included in the introduction section... unless I do EDA here, then probably leave this as its own section



\section{Modeling Background}

This whole section might be able to be merged with the methods section 

\subsection{Occupancy Models}

\noindent{\bfseries free write... this is very messy, basically just word vomit}

\noindent{\bfseries This is all information extracted from WILD 502 course material or things I learned from the class} 

Occupancy is the presence of a particular species on a given site, this may not be the first choice of state variables to ecologists but occupancy studies are useful when there is a large spatial scale or the study is conducted over many years, when abundance or vital rates are hard to measure. Occupancy studies are also useful over capture-recapture methods when individuals cannot be marked or uniquely identified. However, sometimes patterns of species occurrence are of interest, this happens when researchers are interested in the range of a species or the spread of invasion. 

The sampling units for occupancy studies are called 'sites'. We can learn about detection probabilities when multiple site visits are used. Also, when using occupancy models we need to account for imperfect detection because it is possible that the researchers could miss the species even if it is present at the site. $\psi$ represents the occupancy probability, $p_i$ represents the probability of detecting the species on survey $i$ given that the species occupies the site, and $p^* = 1- \prod_{i = 1}^{t} (1-p_i)$ is the probability of detecting the species at least one time given the species occupies the site. 

The assumptions are: 
\begin{itemize}
\item The occupancy state of sites is constant during a single season. 
\item The occupancy probability is constant across sites, or is modeled appropriately using site-level covariates. 
\item The probability of detection given occupancy status is constant across sites, or modeled appropriately using site-level covariates. 
\item The species is not misidentified, no false positives. 
\end{itemize}

As suggested above, site-level covariates can be used to model the occupancy probabilities and the detection probabilities. 

In WILD 502 when talking about multi-season occupancy models, we talked about extirpation and colonization rates, but I think that these could be modeled with a latent variable(s)? I don't think they are of particular interest here. 

In this case, if we consider site to be the lake then we have replication at the site level, but if each site is the sample location within the lake then there is replication for some (very few) on different dates. 

\noindent{\bfseries This is what I used for my project proposal... Mark said this could make it's way into our abstract}

Zebra mussels are a highly invasive species that have several negative impacts, both ecologically and economically, as they suffocate native mussels and cost millions of dollars to remove from man-made structures such as power plants (USGS). Zebra mussels have been established in the United States since the 1980s, and since then "they have spread rapidly throughout the Great Lakes region and into the large rivers of the eastern Mississippi drainage. They have also been found in Texas, Colorado, Utah, Nevada, and California" (USGS). As the threat of zebra mussels spreading into the northwestern United States grows, one concern is that the detection method used to find zebra mussels is not highly effective.

\noindent{\bfseries I have also been working on fitting models to simulated occupancy data} (see code at \url{https://github.com/meaghanwinder/Writing-Project/blob/master/Writing%20Project%20Code.Rmd})

\begin{itemize}
\item constant $\psi = 0.6$ and constant $p = 0.6$ with constant $J = 8$ samples per $M = 10$ sites -- DONE
\item constant $\psi = 0.6$ and constant $p = 0.6$ with $j_i$ samples per $M = 10$ sites -- DONE
\item $\psi$ and $p$ both depend on a single site level covariate, $x$, with $j_i$ samples per $M = 10$ sites
\end{itemize}

$y_{i,j} = \{0, 1\}$ is the binary response for the $j^{th}$ sample within site $i$

If the site is occupied: 
\begin{itemize}
\item $y_{i,j} \sim Bernoulli(p)$
\item $y_i \sim Binomial(J, p)$ where $y_i$ is the total number of detections from the $J$ samples in the $i^{th}$ site 
\end{itemize}

If the site is unoccupied: 
\begin{itemize}
\item $y_{i,j} = 0$ with probability 1, since we assume there are no false detections
\end{itemize}







Things to think about for the simulation or questions I have: 
\begin{itemize}
\item think about how changing the detection probabilities and occupancy probabilities impact the results: 
	\begin{itemize}
	\item High occupancy, high detection
	\item High occupancy, low detection 
	\item Low occupancy, high detection 
	\item Low occupancy, low detection 
	\end{itemize}
\item How do we/can we include sample level covariates to account for sampling effort (number of tows, if available)?
\item If we are getting more data:
	\begin{itemize}
	\item How do we account for multiple sampling seasons (years)? 
	\item How do the assumptions of occupancy models change when we have several sampling years versus only 1
\end{itemize}
\end{itemize}

Additional questions, not directly related to the simulation:
\begin{itemize}
	\item How do we account for this multilevel (for lack of a better word) testing process? For example, sometimes (?) when they fail to find them in the microscope, then they test them using polymerase chain reaction (PCR), and sometimes gene sequencing (? -- looks like it was only used on one observation in the sample data), or scanning electron microscopy (SEM).
	\item One assumption of occupancy models is that there is a non-zero probability of misidentification. I read an article (Denise Holser) that suggests that there might be an issue with that assumption in this situation because there are similar looking organisms that may be present in the waters; she then suggest that the Bureau of Reclamation has attempted to mediate this issue with improved microscopic methods and improved PCR methods. I will look into this some more.   
\end{itemize}

 
\subsection{Bayesian Modeling Background}

\section{Methods}

Package options for Multi-season single-species occupancy models: 
\begin{itemize}
\item nimble.dynamic.occ
\item STAN
\item JAGS
\item wiquid package??
\item Frequentist Options: 
  \begin{itemize}
  \item unmarked
  \item Program MARK
  \end{itemize}
\item write my own package? 
\end{itemize}

\section{Analysis}

Could do EDA here

\section{Conclusion}

\subsection{Further Investigations}

\newpage
\section{References}
\begingroup
\renewcommand{\section}[2]{}%
\begin{flushleft}
\bibliographystyle{apalike}
%\bibliographystyle{acm}
%\bibliographystyle{abbrvnat}
%\bibliographystyle{unsrtnat}
%\bibliographystyle{ACM-Reference-Format}
\bibliography{bibliography}
\end{flushleft}
\endgroup

\newpage
\section{Appendix - R Code}

A script containing all code used for this analysis is available at \\
\centering{\textit{github link here(?)... either that or include all code here}} 

\singlespacing

\begin{knitrout}
\definecolor{shadecolor}{rgb}{0.969, 0.969, 0.969}\color{fgcolor}\begin{kframe}
\begin{alltt}
\hlcom{# SIMULATED DATA 1}
\hlkwd{set.seed}\hlstd{(}\hlnum{1202020}\hlstd{)}
\hlstd{p} \hlkwb{<-} \hlnum{0.6} \hlcom{# constant detection probability... could change to depend on covariates}
\hlstd{psi} \hlkwb{<-} \hlnum{0.6} \hlcom{# constant occupancy probability... could change to depend on covariates }
\hlstd{M} \hlkwb{<-} \hlnum{10} \hlcom{# number of sites}
\hlstd{J} \hlkwb{<-} \hlnum{8} \hlcom{# constant number of samples per site... does not need to be constant}

\hlstd{z} \hlkwb{<-} \hlkwd{rbinom}\hlstd{(M,} \hlnum{1}\hlstd{, psi)} \hlcom{# site occupancy}

\hlstd{y1} \hlkwb{<-} \hlkwd{matrix}\hlstd{(}\hlnum{NA}\hlstd{,} \hlkwc{nrow} \hlstd{= M}\hlopt{*}\hlstd{J,} \hlkwc{ncol} \hlstd{=} \hlnum{4}\hlstd{)}
\hlstd{y1[,} \hlnum{1}\hlstd{]} \hlkwb{<-} \hlkwd{rep}\hlstd{(}\hlnum{1}\hlopt{:}\hlstd{M,} \hlkwc{each}  \hlstd{= J)} \hlcom{# column indicating site}
\hlstd{y1[,} \hlnum{2}\hlstd{]} \hlkwb{<-} \hlkwd{rep}\hlstd{(}\hlnum{1}\hlopt{:}\hlstd{J, M)} \hlcom{# column indicating sample within site}
\hlstd{y1[,} \hlnum{3}\hlstd{]} \hlkwb{<-} \hlkwd{rep}\hlstd{(z,} \hlkwc{each} \hlstd{= J)} \hlcom{# column indicating true site occupancy}

\hlcom{# creates a column of 1's and 0's indicating whether the species was detected}
\hlkwa{for}\hlstd{(i} \hlkwa{in} \hlnum{1}\hlopt{:}\hlstd{(M}\hlopt{*}\hlstd{J))\{}
    \hlstd{y1[i,} \hlnum{4}\hlstd{]} \hlkwb{<-} \hlkwd{rbinom}\hlstd{(}\hlnum{1}\hlstd{,} \hlnum{1} \hlstd{, p}\hlopt{*}\hlstd{y1[i,} \hlnum{3}\hlstd{])} \hlcom{# if the site is not occupied the species cannot be detected}
\hlstd{\}}

\hlkwd{colnames}\hlstd{(y1)} \hlkwb{<-} \hlkwd{c}\hlstd{(}\hlstr{"Site"}\hlstd{,} \hlstr{"Sample"}\hlstd{,} \hlstr{"True Occupancy"}\hlstd{,} \hlstr{"Detected"}\hlstd{)}

\hlkwd{library}\hlstd{(}\hlstr{"car"}\hlstd{)}
\hlkwd{some}\hlstd{(y1)} \hlcom{# view 10 sample rows of the simulated data}

\hlcom{# or}

\hlstd{y2} \hlkwb{<-} \hlkwd{rbinom}\hlstd{(M, J, p}\hlopt{*}\hlstd{z)} \hlcom{# total number of detections for the J samples within each of the M sites, y_i}
\hlcom{# SIMULATED DATA 2}
\hlkwd{set.seed}\hlstd{(}\hlnum{1222020}\hlstd{)}
\hlstd{p} \hlkwb{<-} \hlnum{0.6} \hlcom{# constant detection probability... could change to depend on covariates}
\hlstd{psi} \hlkwb{<-} \hlnum{0.6} \hlcom{# constant occupancy probability... could change to depend on covariates }
\hlstd{M} \hlkwb{<-} \hlnum{10} \hlcom{# number of sites}
\hlstd{J} \hlkwb{<-} \hlkwd{sample}\hlstd{(}\hlnum{1}\hlopt{:}\hlnum{10}\hlstd{, M,} \hlkwc{replace} \hlstd{= T)} \hlcom{# number of times each of the sites were sampled}

\hlstd{z} \hlkwb{<-} \hlkwd{rbinom}\hlstd{(M,} \hlnum{1}\hlstd{, psi)} \hlcom{# site occupancy}

\hlstd{y} \hlkwb{<-} \hlkwd{rep}\hlstd{(}\hlnum{NA}\hlstd{, M)}

\hlkwa{for}\hlstd{(i} \hlkwa{in} \hlnum{1}\hlopt{:}\hlstd{M)\{}
  \hlstd{y[i]} \hlkwb{<-} \hlkwd{rbinom}\hlstd{(}\hlnum{1}\hlstd{, J[i], p}\hlopt{*}\hlstd{z[i])} \hlcom{# total number of detections for the J samples within each of the M sites, y_i}
\hlstd{\}}
\hlcom{#SIMULATED DATA 3}
\hlkwd{set.seed}\hlstd{(}\hlnum{1222020}\hlstd{)}
\hlstd{M} \hlkwb{<-} \hlnum{10} \hlcom{# number of sites}
\hlstd{x} \hlkwb{<-} \hlkwd{runif}\hlstd{(}\hlnum{10}\hlstd{,} \hlnum{0}\hlstd{,} \hlnum{10}\hlstd{)}
\hlstd{beta1.true} \hlkwb{<-} \hlnum{2}
\hlstd{beta2.true} \hlkwb{<-} \hlnum{0.5}
\hlstd{beta3.true} \hlkwb{<-} \hlnum{0.5}
\hlstd{beta4.true} \hlkwb{<-} \hlnum{0.2}
\hlstd{p} \hlkwb{<-} \hlkwd{exp}\hlstd{(beta1.true} \hlopt{-} \hlstd{beta2.true}\hlopt{*}\hlstd{x)}\hlopt{/}\hlstd{(}\hlnum{1} \hlopt{+} \hlkwd{exp}\hlstd{(beta1.true} \hlopt{-} \hlstd{beta2.true}\hlopt{*}\hlstd{x))} \hlcom{# detection probability depends on site covariate x}
\hlstd{psi} \hlkwb{<-} \hlkwd{exp}\hlstd{(beta3.true} \hlopt{+} \hlstd{beta4.true}\hlopt{*}\hlstd{x)}\hlopt{/}\hlstd{(}\hlnum{1} \hlopt{+} \hlkwd{exp}\hlstd{(beta3.true} \hlopt{+} \hlstd{beta4.true}\hlopt{*}\hlstd{x))} \hlcom{# occupancy probability depends on site covariate x}
\hlstd{J} \hlkwb{<-} \hlkwd{sample}\hlstd{(}\hlnum{1}\hlopt{:}\hlnum{10}\hlstd{, M,} \hlkwc{replace} \hlstd{= T)} \hlcom{# number of times each of the sites were sampled}

\hlstd{z} \hlkwb{<-} \hlkwd{rep}\hlstd{(}\hlnum{NA}\hlstd{, M)}
\hlstd{y} \hlkwb{<-} \hlkwd{rep}\hlstd{(}\hlnum{NA}\hlstd{, M)}

\hlkwa{for}\hlstd{(i} \hlkwa{in} \hlnum{1}\hlopt{:}\hlstd{M)\{}
  \hlstd{z[i]} \hlkwb{<-} \hlkwd{rbinom}\hlstd{(}\hlnum{1}\hlstd{,} \hlnum{1}\hlstd{, psi[i])} \hlcom{#site occupancy}
  \hlstd{y[i]} \hlkwb{<-} \hlkwd{rbinom}\hlstd{(}\hlnum{1}\hlstd{, J[i], p[i]}\hlopt{*}\hlstd{z[i])} \hlcom{# total number of detections for the J samples within each of the M sites, y_i}
\hlstd{\}}
\hlkwd{library}\hlstd{(}\hlstr{"readxl"}\hlstd{)}
\hlstd{USBR.login} \hlkwb{<-} \hlkwd{read_excel}\hlstd{(}\hlstr{"C:/Users/mwind/OneDrive/Writing-Project/USBR_2018_data_sample.xlsx"}\hlstd{,} \hlkwc{sheet} \hlstd{=} \hlnum{1}\hlstd{)}
\hlstd{USBR.sampleprep} \hlkwb{<-} \hlkwd{read_excel}\hlstd{(}\hlstr{"C:/Users/mwind/OneDrive/Writing-Project/USBR_2018_data_sample.xlsx"}\hlstd{,} \hlkwc{sheet} \hlstd{=} \hlnum{2}\hlstd{)}
\hlstd{USBR.microscopy} \hlkwb{<-} \hlkwd{read_excel}\hlstd{(}\hlstr{"C:/Users/mwind/OneDrive/Writing-Project/USBR_2018_data_sample.xlsx"}\hlstd{,} \hlkwc{sheet} \hlstd{=} \hlnum{3}\hlstd{)}
\hlstd{USBR.PCR} \hlkwb{<-} \hlkwd{read_excel}\hlstd{(}\hlstr{"C:/Users/mwind/OneDrive/Writing-Project/USBR_2018_data_sample.xlsx"}\hlstd{,} \hlkwc{sheet} \hlstd{=} \hlnum{4}\hlstd{)}

\hlkwd{unique}\hlstd{(USBR.sampleprep}\hlopt{$}\hlstd{`Water Name`)} \hlcom{# number of lakes (sites) in the 2018 sample data}
\end{alltt}
\end{kframe}
\end{knitrout}


\lstset{
	basicstyle=\ttfamily\scriptsize,
	columns=fullflexible,
	frame=single,
	breaklines=true,
	postbreak=\mbox{\textcolor{red}{$\hookrightarrow$}\space},
	keepspaces = TRUE,
}
%\begin{lstlisting}
%\end{lstlisting}
\end{document}
              
